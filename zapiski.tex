%--------------------
% Packages
% -------------------
\documentclass[11pt,a4paper]{article}
\usepackage[utf8x]{inputenc}

\usepackage[pdftex]{graphicx} % Required for including pictures
\usepackage[pdftex,linkcolor=black,pdfborder={0 0 0}]{hyperref} % Format links for pdf
\usepackage{calc} % To reset the counter in the document after title page
\usepackage{enumitem} % Includes lists

\usepackage{ amssymb } % extra math symbols
\usepackage{ amsmath } % extra math symbols
\usepackage{ dsfont } % font za množice
% tabele
\usepackage{array}
\usepackage{wrapfig}
\usepackage{multirow}
\usepackage{tabularx}

\frenchspacing % No double spacing between sentences
\setlength{\parindent}{0pt}
\setlength{\parskip}{0.1em}

\usepackage[a4paper, lmargin=0.1666\paperwidth, rmargin=0.1666\paperwidth, tmargin=0.1111\paperheight, bmargin=0.1111\paperheight]{geometry} %margins

\usepackage{lipsum} % Used for inserting dummy 'Lorem ipsum' text into the template

\begin{document} 

\section*{Integral s parametrom}
Naj bo $D = [a,b]\times [c,d]$ in $f: D \to \mathbb{R}$ dana funkcija. Integralu
\[F(y) = \int_a^b f(x,y) dx \]
pravimo integral s parametrom.

\subsubsection*{Zveznost integrala s parametrom}
Če je $f: [a,b]\times [c,d] \to \mathbb{R}$ zvezna, je $F$ zvezna na $[c,d]$.

\subsubsection*{Odvajnaje pod integralom}
Naj bo $f$ zvezna nad $D$ in $f_y$ zvezna nad $D$. Potem je 
\[F'(y) = \int_a^b f_y(x,y) dx \]

\subsubsection*{Odvajnaje integrala}
Naj bosta $u,v: [c,d] \to [a,b]$ in $u,v \in \mathcal{C}^1([c,d])$,\\
$f \in \mathcal{C}([a,b]\times [c,d])$, $f_y \in \mathcal{C}([a,b]\times [c,d])$

Definirajmo: \[F(y) = \int_{u(y)}^{v(y)} f(x,y) dx\]
Potem je $F$ odvedljiva in \[ F'(y) = \int_{u(y)}^{v(y)} f_y(x,y) dx + f\left(v(y),y\right)v'(y) - f\left(u(y),y\right)u'(y)\]

\subsubsection*{Dvokratno integriranje funkcije}
Naj bo $f \in \mathcal{C}([a,b]\times [c,d])$. Definirajmo
\begin{equation*}
    \begin{aligned}
        F(y) &= \int_a^b f(x,y) dx  & G(x) &= \int_c^d f(x,y) dy 
    \end{aligned}
\end{equation*}
Potem je
\[\int_a^b G(x) dx = \int_c^d F(y) dy = \iint_D f(x,y) dxdy\]

\subsection*{Integrali na neomejenih območjih}
Naj bo $f \in \mathcal{C}([a,\infty] \times [c,d])$:
\[F(y) := \int_{a}^{\infty}f(x,y) dx = \lim_{b \to \infty} \int_{a}^{b}f(x,y) dx \]

\subsubsection*{Enakomerna konvergenca funkcij}
Funkcijsko zaporedje $F_b$ enakomerno konvergira proti $F$, če za $\forall \varepsilon > 0 : \exists B : b > B :$
\[ |F_b(y)-F(y)| < \varepsilon; \quad \forall y \in [c,d]\]
Če to prepišemo v obliki integrala, dobimo:
\[ |F_b(y) - F(y)| = \left|\int_a^b f(x,y)dx - \int_a^{\infty} f(x,y)dx \right| = \left|\int_b^{\infty} f(x,y)dx \right| < \varepsilon; \quad \forall y \in [c,d]\]

Integral s parametrom je enakomerno konvergenten na $[c,d]$, če za $\forall \varepsilon > 0 : \exists B : \forall b > B :$
\[ \left|\int_b^{\infty} f(x,y)dx \right| < \varepsilon \]

\emph{Posledica:} Če je za $f \in \mathcal{C}([a,\infty] \times [c,d])$ integral s parametrom $F(x) = \int_a^{\infty} f(x,y) dx$ enakomerno konvergenten na $[c,d]$, je $F$ zvezna na $[c,d]$.

\subsubsection*{Menjava vrstnega reda integracije}
Naj bo $f \in \mathcal{C} ([a,\infty] \times [c,d])$ in $ F(y) := \int_a^{\infty} f(x,y) dx$ enakomerno konvergentna. Definirajmo še $G(x) = \int_c^d f(x,y) dy$. Potem velja
\[\int_c^d F(y) dy = \int_a^{\infty} G(x) dx  \]
oziroma
\[\int_c^d dy \int_a^{\infty} f(x,y) dx = \int_a^{\infty} dx \int_c^d f(x,y) dy \]

\subsubsection*{Odvajanje integrala na neomejenih območjih}
Naj bosta $f,f_y \in \mathcal{C}([a,\infty]\times [c,d])$, $F(y) := \int_a^{\infty} f(x,y) dx$ obstaja za $\forall y \in [c,d]$ 
in $G(y) := \int_a^{\infty} f_y(x,y) dx$ enakomerno konvergira na $[c,d]$.

Potem je $F$ odvedljiva in $F'(y) = G(y)$.

\subsection*{Funkcija $\Gamma$}
Funkcija gama je definirana s predpisom
\[ \Gamma(s) = \int_0^{\infty} x^{s-1} e^{-x} dx, \qquad \forall s > 0 \]

Lastnosti funkcije gama:
\begin{enumerate}
    \item $\Gamma(1) = 1$
    \item $\Gamma(s+1) = s \Gamma(s)$
    \item $\Gamma(\frac{1}{2}) = \sqrt{\pi}$
    \item $\Gamma(n+1) = n!\qquad n\in \mathbb{R}$
\end{enumerate}

\subsection*{Funkcija $B$}
Funkciaj beta je definirana kot
\[B(p,q) = \int_0^1 x^{p-1} (1-x)^{q-1} dx, \qquad \forall p,q > 0\]

Lastnosti funkcije beta:
\begin{enumerate}
    \item simetričnost: $B(p,q) = B(q,p)$
    \item $\displaystyle B(p,q) = \int_0^{\infty} \frac{u^{p-1}}{(1+u)^{p+q}}du $
    \item $\displaystyle B(p,q) = \frac{\Gamma(p) \Gamma(q)}{\Gamma(p+q)}$
    \item $\displaystyle \int_0^{\frac{\pi}{2}} (\sin x)^{2p-1} (\cos x)^{2q-1} = \frac{1}{2} B(p,q)$
\end{enumerate}


\section*{Dvojni in trojni integral}
    
\end{document}